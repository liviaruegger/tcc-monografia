% Author: Nelson Lago
% This file is distributed under the MIT Licence

%%%%%%%%%%%%%%%%%%%%%%%%%%%%%%%%%%%%%%%%%%%%%%%%%%%%%%%%%%%%%%%%%%%%%%%%%%%%%%%%
%%%%%%%%%%%%%%%%%%%%%%%%%%%%%%%%% PREÂMBULO %%%%%%%%%%%%%%%%%%%%%%%%%%%%%%%%%%%%
%%%%%%%%%%%%%%%%%%%%%%%%%%%%%%%%%%%%%%%%%%%%%%%%%%%%%%%%%%%%%%%%%%%%%%%%%%%%%%%%

% A língua padrão é a última da lista
\documentclass[a1paper,brazilian,english]{article}

% Vários pacotes e opções de configuração genéricos
\usepackage{imegoodies}
\usepackage[poster,hidelinks]{imelooks}
% \tcbposterset{fontsize = 32pt} % default, mude se necessário

% Diretórios onde estão as figuras; com isso, não é necessário (mas
% é permitido) colocar o caminho completo em \includegraphics. Note
% que a extensão nunca é necessária (mas é permitida), ou seja, o
% resultado é o mesmo com "\includegraphics{figuras/foto.jpeg}",
% "\includegraphics{foto.jpeg}", "\includegraphics{figuras/foto}"
% ou "\includegraphics{foto}".
\graphicspath{{figuras/},{fig/},{logos/},{img/},{images/},{imagens/}}

% Comandos rápidos para mudar de língua:
% \en -> muda para o inglês
% \br -> muda para o português
% \texten{blah} -> o texto "blah" é em inglês
% \textbr{blah} -> o texto "blah" é em português
\babeltags{br = brazilian, en = english}


%%%%%%%%%%%%%%%%%%%%%%%%%%%%%%%%%%%%%%%%%%%%%%%%%%%%%%%%%%%%%%%%%%%%%%%%%%%%%%%%
%%%%%%%%%%%%%%%%%%%%%%%%%%%%%%%%%% METADADOS %%%%%%%%%%%%%%%%%%%%%%%%%%%%%%%%%%%
%%%%%%%%%%%%%%%%%%%%%%%%%%%%%%%%%%%%%%%%%%%%%%%%%%%%%%%%%%%%%%%%%%%%%%%%%%%%%%%%

% O arquivo com os dados bibliográficos para biblatex; você pode usar
% este comando mais de uma vez para acrescentar múltiplos arquivos
\addbibresource{bibliografia.bib}

% Este comando permite acrescentar itens à lista de referências sem incluir
% uma referência de fato no texto (pode ser usado em qualquer lugar do texto)
%\nocite{bronevetsky02,schmidt03:MSc, FSF:GNU-GPL, CORBA:spec, MenaChalco08}
% Com este comando, todos os itens do arquivo .bib são incluídos na lista
% de referências
%\nocite{*}


%%%%%%%%%%%%%%%%%%%%%%%%%%%%%%%%%%%%%%%%%%%%%%%%%%%%%%%%%%%%%%%%%%%%%%%%%%%%%%%%
%%%%%%%%%%%%%%%%%%%%%%%%%%%%%%% INÍCIO DO POSTER %%%%%%%%%%%%%%%%%%%%%%%%%%%%%%%
%%%%%%%%%%%%%%%%%%%%%%%%%%%%%%%%%%%%%%%%%%%%%%%%%%%%%%%%%%%%%%%%%%%%%%%%%%%%%%%%


% Existem várias packages para criar pôsteres com LaTeX (a0poster, baposter,
% tikzposter, sciposter...). As mais comuns atualmente são beamerposter
% e tcolorbox (com sua biblioteca "poster"). Ambas funcionam muito bem;
% beamerposter é mais familiar (ela simplesmente utiliza beamer com alguns
% ajustes no tamanho das fontes e do papel), mas com tcolorbox o alinhamento
% vertical dos elementos é MUITO mais simples, e esta é a solução adotada
% aqui. Vale muito a pena ler a documentação com "texdoc tcolorbox" e
% "texdoc tcolorbox-tutorial-poster".

% Um pôster com tcolorbox é composto por blocos (posterboxes) coloridos
% de tamanho variável; cada bloco pode conter textos ou imagens e um
% título opcional. O pôster utiliza uma grade de dimensões definidas em
% \begin{tcposter} com "rows=" e "columns=" para fazer o alinhamento:
% para cada posterbox, podemos dizer "row=X, column=Y" para definir sua
% posição. Além disso, podemos dizer "span=A, rowspan=B" para fixar
% seu tamanho. Sem "span" e "rowspan", uma posterbox tem pelo menos o
% tamanho de uma célula da grade, mas se seu tamanho natural for maior
% ela extrapola esse tamanho. "span" e "rowspan" podem ser números
% não-inteiros (como 0.8 ou 1.4).
%
% "\begin{posterbox}" recebe um conjunto de parâmetros opcional e um
% conjunto de parâmetros obrigatório:
%
% "\begin{posterbox}[opcional]{obrigatório}".
%
% O conjunto de parâmetros opcional é onde inserimos os parâmetros comuns
% de tcolorbox, como "adjusted title", "coltext", "titlerule" etc.; o
% conjunto de parâmetros obrigatório é usado para determinar as dimensões
% e a posição da posterbox, ou seja, as opções "name", "column", "below",
% "span" etc.
%
% ALINHAMENTO HORIZONTAL
%
% É possível definir um poster com 2 colunas e fazer algo como
%
% \posterbox{column=1, span=1.3}{blah}
% \posterbox{column*=2, span=0.7}{blah}
%
% A segunda posterbox será alinhada à direita ("column*="), então as
% duas serão colocadas lado-a-lado sem sobreposições.
%
% Na prática, no entanto, é mais fácil fazer como no exemplo abaixo:
% definimos que o poster tem 12 colunas, o que nos permite dividir
% sua largura em 2, 3, 4 ou 6 colunas iguais ou diferentes (como
% 1/2 + 1/2, 2/3 + 1/3, 1/4 + 1/4 + 1/2, 1/4 + 1/6 + 1/4 + 1/3 etc).
%
% ALINHAMENTO VERTICAL
%
% Embora seja possível alinhar as posterboxes em função da grade na
% vertical, uma outra possibilidade é utilizar "above", "below" e
% "between", como no exemplo abaixo: basta associar um nome "blah" a
% uma determinada posterbox e, em outra, dizer "below=blah". Lembre-se
% que a posterbox de nome "blah" deve ser definida *antes* que outra
% possa fazer referência a ela. Também é possível fazer "below=top",
% "above=bottom" etc. A opção "equal height group" também é muito útil.
% Nada impede que você use estratégias de alinhamento diferentes para
% cada posterbox.

% Este modelo define a opção "smallmargins", que diminui a distância
% entre o conteúdo de uma posterbox e suas bordas. Use com parcimônia!

\begin{document}

% Em um poster não há \maketitle

\begin{tcbposter}[
  poster = {
    %showframe, % muito útil durante a preparação do poster
    rows = 6,
    columns = 12,
    colspacing = 1.2cm,
    rowspacing = .8cm,
  },
]


%%%%%%%% Título e autoria %%%%%%%%

\posterbox[titlebox]{name=titlebox, below=top, column=1, span=12}{
    \huge
    Contribuindo com o QGIS\par
    \vspace{6pt}
    \large
    Experiências no desenvolvimento de software livre para Sistemas de Informação Geográfica
}

\posterbox[footerbox]{name=footerbox, above=bottom, column=1, span=12}{
    \large
    Ana Lívia Rüegger Saldanha\par
    \vspace{2pt}
    \small
    Orientadora: Juliana Orro Marquez (UNIFESP)\par
    Coorientador: Paulo Meirelles (IME-USP)\par
    \vspace{6pt}
    \footnotesize\rmfamily
    \textcolor{imesoftblue!30!white}
      {Trabalho de Formatura Supervisionado Voltado à Extensão (MAC0500)}\relax
    \footimage{%
    \includegraphics[width=13cm]{ime-logo}%
    \hspace{2cm}%
    \includegraphics[height=5cm]{figuras/qr-github-footer}%
    }
}


%%%%%%%% Introdução %%%%%%%%

\posterbox[adjusted title = Proposta, equal height group = toprow]
          {name=resumo, below=titlebox, column=1, span=4}{
    \small
    Contribuir com o QGIS, um dos principais projetos de software livre para Sistemas de Informação Geográfica (SIG), a partir da identificação de uma necessidade real de usuários da plataforma na universidade e comunidade científica.
}

\posterbox[adjusted title = Contexto, equal height group = toprow]
          {name=contexto, below=titlebox, column=5, span=4}{
    \small
    \begin{itemize}
        \item Crescente uso de \textbf{Gêmeos Digitais} e ambientes imersivos (VR) para gestão pública e análise de riscos (ex: inundações)
        \item Necessidade de converter dados geoespaciais reais (MDE em GeoTIFF) para ambientes 3D virtuais
    \end{itemize}
}

\posterbox[adjusted title = Problemas, equal height group = toprow]
          {name=problemas, below=titlebox, column=9, span=4}{
    \small
    \begin{itemize}
        \item Processo atual: \textbf{manual}, exige linhas de comando complexas com múltiplos parâmetros técnicos e conversões intermediárias
        \item Barreira técnica para pesquisadores não especializados em computação
    \end{itemize}
}

\posterbox[adjusted title = Objetivo]
          {name=objetivo, below=resumo, column=1, span=12}{
    \small
    Desenvolver uma ferramenta para conversão de Modelos Digitais de Elevação (MDE) no formato GeoTIFF para o formato RAW compatível com Unity, a fim de facilitar a utilização de dados reais na construção de terrenos tridimensionais.
}


%%%%%%%% Prova de Conceito (definida primeiro para referência) %%%%%%%%

% Constantes para imagens
\newcommand{\alturaimagem}{9cm}       % Altura padrão das imagens na prova de conceito
\newcommand{\separacaoimagens}{1.5cm} % Espaçamento entre as imagens (entre bordas)
\newcommand{\espacolegenda}{0.3cm}    % Espaçamento entre imagem e legenda

% Definir "prova" primeiro para poder ser referenciada
\posterbox[adjusted title = Prova de Conceito]
          {name=prova, above=footerbox, column=1, span=12}{
    \centering
    \vspace{8pt}
    \begin{tikzpicture}[auto]
        \usetikzlibrary{positioning}
        % Nó 0: Plugin Instalado
        \node[inner sep=0pt] (plugin_instalado) {
            \includegraphics[height=\alturaimagem]{figuras/poster/0_plugin_info}
        };
        \node[below=\espacolegenda of plugin_instalado, font=\small\bfseries] {Plugin Instalado};
        
        % Nó 1: GeoTIFF
        \node[inner sep=0pt, right=\separacaoimagens of plugin_instalado] (geotiff) {
            \includegraphics[height=\alturaimagem]{figuras/poster/1_geotiff}
        };
        \node[below=\espacolegenda of geotiff, font=\small\bfseries] {MDE em GeoTIFF};
        
        % Nó 2: Plugin QGIS
        \node[inner sep=0pt, right=\separacaoimagens of geotiff] (plugin) {
            \includegraphics[height=\alturaimagem]{figuras/poster/2_plugin_dialog}
        };
        \node[below=\espacolegenda of plugin, font=\small\bfseries] {Processamento};
        
        % Nó 3: Logs de conversão
        \node[inner sep=0pt, right=\separacaoimagens of plugin] (logs) {
            \includegraphics[height=\alturaimagem]{figuras/poster/3_logs}
        };
        \node[below=\espacolegenda of logs, font=\small\bfseries] {Logs de conversão};

        % Nó 4: Terreno 3D
        \node[inner sep=0pt, right=\separacaoimagens of logs] (terreno) {
            \includegraphics[height=\alturaimagem]{figuras/poster/4_terrain}
        };
        \node[below=\espacolegenda of terreno, font=\small\bfseries] {Terreno 3D (Unity)};
        
        % Setas finas com numeração
        \draw[->, thin, >=stealth, line width=0.5pt] (plugin_instalado.east) -- node[above, font=\tiny] {\textbf{1}} (geotiff.west);
        \draw[->, thin, >=stealth, line width=0.5pt] (geotiff.east) -- node[above, font=\tiny] {\textbf{2}} (plugin.west);
        \draw[->, thin, >=stealth, line width=0.5pt] (plugin.east) -- node[above, font=\tiny] {\textbf{3}} (logs.west);
        \draw[->, thin, >=stealth, line width=0.5pt] (logs.east) -- node[above, font=\tiny] {\textbf{4}} (terreno.west);
    \end{tikzpicture}
    
    \vspace{10pt}
    \small
    (1) Após instalar o plugin, o usuário seleciona camada raster no QGIS →
    (2) Através de uma interface simplificada, o usuário define o arquivo de saída →
    (3) Plugin processa automaticamente (recorte, normalização 16-bit, detecção de padding, cálculo de dimensões), gera arquivo .raw compatível e apresenta logs de conversão, incluindo parâmetros sugeridos para importação no Unity →
    (4) O usuário importa o arquivo .raw no Unity, com as configurações sugeridas, e visualiza o terreno 3D.
    \vspace{8pt}
}


%%%%%%%% Processamento e Próximos passos %%%%%%%%

\posterbox[adjusted title = Próximos passos]
          {name=proximos-passos, above=prova, column=7, span=6}{
    \small
    Publicar o plugin no \textbf{repositório oficial de plugins do QGIS}, tornando-o disponível para toda a comunidade de usuários da plataforma através da interface nativa de gerenciamento de plugins.
}

\posterbox[adjusted title = Processamento]
          {name=processamento, between=objetivo and proximos-passos, column=7, span=6}{
    \small
    Utilizando as ferramentas da \textbf{GDAL/OGR}, o plugin automatiza o processo de conversão para o formato RAW compatível com Unity, incluindo:
    \begin{itemize}
        % \item Recorte e resolução compatível com Unity;
        \item Normalização 16-bit preservando a faixa de elevação;
        \item Detecção e exclusão de padding/NoData;
        \item Cálculo de dimensões corretas (X, Y, Z) considerando sistemas de coordenadas (geográfico vs. projetado).
    \end{itemize}
}


%%%%%%%% Solução %%%%%%%%

\posterbox[adjusted title = Solução]
          {name=solucao1, between=objetivo and prova, column=1, span=6}{
    \vspace{12pt}
    \normalsize
    \begin{minipage}{0.08\textwidth}
        \centering
        \includegraphics[width=\textwidth]{figuras/icon}
    \end{minipage}%
    \hspace{1cm}%
    \begin{minipage}{0.85\textwidth}
        \textbf{Unity Terrain Exporter\\}
        Plugin QGIS (Processing Toolbox)
    \end{minipage}

    \vspace{22pt}
    \small
    Uma das maiores vantagens do QGIS é sua \textbf{arquitetura extensível} por meio de \textbf{plugins}, que permite a criação de novas funcionalidades para o software de maneira flexível e personalizada, porém sem a necessidade de alterar o código-fonte do core do software.
    
    \vspace{16pt}
    \begin{itemize}
        \item Desenvolvido em Python usando PyQGIS API
        \item Utiliza a Geospatial Data Abstraction Library (GDAL/OGR) para processamento de dados geoespaciais
        \item Integrado ao QGIS Processing Framework
        \item Licenciado sob GPL-3.0
    \end{itemize}
    
    \vspace{12pt}
    \centering
    %% Logo do QGIS
    \begin{minipage}[t]{0.3\textwidth}
        \centering
        \raisebox{0.45cm}{\includegraphics[height=1.2cm]{figuras/qgis-logo}}
    \end{minipage}%
    \raisebox{0.6cm}{$\rightarrow$}%
    %% Logo da GDAL
    \begin{minipage}[t]{0.3\textwidth}
        \centering
        \raisebox{-0.4cm}{\includegraphics[height=3cm]{figuras/gdal-logo}}
    \end{minipage}%
    \raisebox{0.6cm}{$\rightarrow$}%
    %% Logo do Unity
    \begin{minipage}[t]{0.3\textwidth}
        \centering
        \includegraphics[height=2cm]{figuras/unity-logo}
    \end{minipage}
}

\end{tcbposter}

\end{document}
