%!TeX root=../tese.tex
%("dica" para o editor de texto: este arquivo é parte de um documento maior)
% para saber mais: https://tex.stackexchange.com/q/78101

% As palavras-chave são obrigatórias, em português e em inglês, e devem ser
% definidas antes do resumo/abstract. Acrescente quantas forem necessárias.
\palavraschave{QGIS, Sistemas de Informação Geográfica, Software Livre, GeoTIFF, Unity, Modelos Digitais de Elevação, Gêmeos Digitais}

\keywords{QGIS, Geographic Information Systems, Open Source, GeoTIFF, Unity, Digital Elevation Models, Digital Twins}

% O resumo é obrigatório, em português e inglês. Estes comandos também
% geram automaticamente a referência para o próprio documento, conforme
% as normas sugeridas da USP.
\resumo{
    A proposta deste trabalho de conclusão de curso, no contexto da extensão universitária, é contribuir com o QGIS, um dos principais projetos de software livre para Sistemas de Informação Geográfica (SIG), identificando um novo potencial caso de uso a partir de uma necessidade real de usuários da plataforma na universidade e comunidade científica. O trabalho documenta o estudo da estrutura do projeto e sua comunidade de desenvolvimento, as especificações técnicas do problema abordado e o projeto de uma nova funcionalidade para o QGIS: um conversor de Modelos Digitais de Elevação (MDE) em formato \emph{GeoTIFF} para o formato \emph{Raw} compatível com a plataforma de desenvolvimento 3D Unity, a fim de facilitar a construção de terrenos virtuais realistas em ambientes imersivos para aplicações diversas, como simulações ambientais, realidade virtual e Gêmeos Digitais.
}

\abstract{
    The proposal of this capstone project, in the context of university extension, is to contribute to QGIS, one of the main free and open source projects for Geographic Information Systems (GIS), by identifying a new potential use case from a real need of platform users in the university and scientific community. The work documents the study of the project structure and its development community, the technical specifications of the addressed problem, and the design of a new functionality for QGIS: a converter of Digital Elevation Models (DEM) in GeoTIFF format to the Raw format compatible with the Unity 3D development platform, in order to facilitate the construction of realistic virtual terrains in immersive environments for diverse applications, such as environmental simulations, virtual reality, and Digital Twins.
}
