%!TeX root=../tese.tex
%("dica" para o editor de texto: este arquivo é parte de um documento maior)
% para saber mais: https://tex.stackexchange.com/q/78101

\chapter{Estudo de caso: conversão de arquivos de mapa de elevação}
\label{cap:estudo-caso}

\section{Metodologia}
\label{sec:metodologia}

% TODO: Conteúdo a ser desenvolvido

\section{Especificações técnicas do problema}
\label{sec:especificacoes}

\subsection{Soluções alternativas}
\label{subsec:solucoes-alternativas}

% TODO: Conteúdo a ser desenvolvido

\section{Proposta}
\label{sec:proposta}

Diante do obstáculo técnico que a conversão manual de dados de elevação representa, um processo que exige familiaridade com ferramentas de linha de comando (CLI) como a GDAL e conhecimento sobre especificações de formatos de arquivo, propõe-se o desenvolvimento de um plugin para o QGIS denominado provisoriamente de ``GeoTIFF to Unity Terrain''.

Este plugin visa preencher a lacuna de interoperabilidade entre o ambiente de análise geoespacial (QGIS) e a plataforma de desenvolvimento de ambientes imersivos (Unity). A solução objetiva se apresentar como uma ferramenta com interface gráfica simples e específica (em oposição à funcionalidade existente de \emph{Translate}, apresentada no item 3.3.2), integrada diretamente ao QGIS. O objetivo principal é automatizar o fluxo de trabalho de conversão de um MDE no formato GeoTIFF para um arquivo de terreno no formato RAW (.raw), em conformidade com as especificações exigidas pelo motor de jogos Unity (16-bit, unsigned integer, com ordem de bytes selecionável).

Ao encapsular a complexidade do processo de conversão, o plugin atuará como uma ponte, permitindo que pesquisadores, geógrafos, urbanistas e outros profissionais não especializados em computação possam gerar terrenos 3D realistas a partir de dados geoespaciais com autonomia e eficiência. A proposta não apenas soluciona uma necessidade direta do estudo de caso de doutorado em questão, mas também contribui para o ecossistema QGIS com uma ferramenta de nicho que fortalece a integração entre geotecnologias e a indústria de simulação e visualização em tempo real.

\subsection{Requisitos funcionais}
\label{subsec:requisitos-funcionais}

Os requisitos funcionais especificam o que o sistema deve ser capaz de fazer. Para capturar estes requisitos a partir da perspectiva do usuário final, foi utilizada a metodologia de Histórias de Usuário (\emph{User Stories}), que descrevem uma funcionalidade do software de forma simples e direta.

\begin{table}[h]
    \centering
    \caption{História principal (Épico)}
    \label{tab:epico}
    \begin{tabular}{p{\textwidth}}
        \toprule
        Como uma pesquisadora que trabalha com ambientes digitais imersivos modelados no Unity, eu quero converter um arquivo GeoTIFF de elevação para um formato de terreno compatível com o Unity diretamente no QGIS, para que eu possa criar Modelos Digitais de Elevação (MDE) de forma rápida e sem a necessidade de conhecimento técnico avançado. \\
        \bottomrule
    \end{tabular}
\end{table}

A partir deste épico, derivam-se as seguintes histórias de usuário detalhadas, que se traduzem em requisitos funcionais específicos para o plugin:

\begin{table}[h]
    \centering
    \caption{Histórias de usuário detalhadas}
    \label{tab:user-stories}
    \begin{tabular}{p{5cm}p{9cm}}
        \toprule
        \textbf{História}                                & \textbf{Descrição}                                                                                                                                                                                                      \\
        \midrule
        \textit{Seleção do Dado de Entrada}              & Como usuário, eu quero selecionar uma camada raster (GeoTIFF) que já está aberta no meu projeto QGIS, para que eu não precise navegar por pastas do sistema novamente.                                                  \\
        \textit{Definição do Arquivo de Saída}           & Como usuário, eu quero especificar o local e o nome do arquivo .raw que será gerado, para que eu possa organizar meus arquivos de projeto de maneira controlada.                                                        \\
        \textit{Configuração Simplificada de Parâmetros} & Como usuário, eu quero que o plugin assuma as configurações técnicas complexas por padrão (como a conversão para 16-bit unsigned integer), para que eu não precise entender os detalhes do formato de arquivo do Unity. \\
        \textit{Execução e Feedback}                     & Como usuário, eu quero clicar em um único botão para iniciar a conversão e receber uma notificação clara de sucesso ou de erro ao final do processo, para que eu saiba se o arquivo foi gerado corretamente.            \\
        \bottomrule
    \end{tabular}
\end{table}

Com base nas histórias de usuário apresentadas, os seguintes requisitos funcionais (RF) são definidos para o plugin:

\begin{enumerate}
    \item[RF01] O sistema deve permitir que o usuário selecione uma camada raster ativa no painel de camadas do QGIS como dado de entrada.
    \item[RF02] O sistema deve apresentar uma interface gráfica para que o usuário possa definir o caminho e o nome do arquivo de saída com a extensão .raw.
    \item[RF03] O sistema deve, internamente, realizar a conversão do dado de entrada para o formato de 16-bit sem sinal (unsigned integer), que é o padrão exigido pelo Unity.
    \item[RF04] Caso a camada de entrada contenha dados em ponto flutuante, o sistema deve aplicar uma normalização para escalar os valores de elevação para o intervalo de 0 a 65535.
    \item[RF05] A interface deve oferecer ao usuário a opção de selecionar a ordem de bytes (byte order/endianness) do arquivo de saída (ex: ``Intel/Windows'' ou ``Mac'').
    \item[RF06] O sistema deve exibir uma mensagem de confirmação ao usuário após a conclusão bem-sucedida da conversão.
    \item[RF07] Em caso de falha (ex: camada de entrada inválida), o sistema deve exibir uma mensagem de erro informativa e de fácil compreensão.
    \item[RF08] A funcionalidade de conversão deve ser acionada por meio de um botão (ex: ``Converter'' ou ``Executar'').
\end{enumerate}

\subsection{Ferramentas existentes e arquitetura alternativa}
\label{subsec:ferramentas-existentes}

% TODO: Existe uma ferramenta que usa diretamente GDAL_translate, genérica - linkar as discussões do github

% TODO: Diante da existência de tal funcionalidade, foi considerada também a possibilidade de integrar a solução a esse menu. No entanto, entendemos que a manutenção de uma funcionalidade genérica, com a possibilidade de adição de parâmetros opcionais (...)

\subsection{Solução adotada}
\label{subsec:solucao-adotada}

A solução adotada foi a criação de um novo plugin para o QGIS~\citep{qgisPlugins}, denominado \emph{Unity Terrain Exporter}. Esta escolha se justifica pela arquitetura extensível do QGIS, que permite a criação de novas funcionalidades por meio de plugins de maneira flexível e personalizada, sem a necessidade de alterar o código-fonte do núcleo do software.

O plugin foi desenvolvido em Python utilizando a API PyQGIS, que fornece uma interface de programação completa para interagir com as funcionalidades do QGIS. Para o processamento dos dados geoespaciais, a solução utiliza a \emph{Geospatial Data Abstraction Library} (GDAL/OGR), uma biblioteca amplamente utilizada e confiável para manipulação de dados geoespaciais. A integração com o \emph{QGIS Processing Framework} permite que o plugin seja acessível através da interface nativa de processamento do QGIS, facilitando sua descoberta e uso pelos usuários.

O projeto é licenciado sob a GPL-3.0, alinhando-se com a filosofia de software livre do QGIS e garantindo que a solução permaneça aberta e acessível à comunidade. O código-fonte está disponível publicamente no repositório GitHub: \url{https://github.com/liviaruegger/unity-terrain-exporter}.

A arquitetura da solução conecta três ecossistemas principais: o QGIS, como ambiente de análise geoespacial; a GDAL, como biblioteca de processamento de dados; e o Unity, como plataforma de destino para visualização 3D.

\subsection{Prova de conceito}
\label{subsec:prova-conceito}

Uma prova de conceito funcional foi desenvolvida e está disponível publicamente no repositório GitHub: \url{https://github.com/liviaruegger/unity-terrain-exporter}. Esta implementação demonstra a viabilidade técnica da solução proposta e valida os requisitos funcionais estabelecidos.

O fluxo de trabalho da prova de conceito pode ser dividido em quatro etapas principais, ilustradas na Figura~\ref{fig:prova-conceito}:

\begin{enumerate}
    \item \textbf{Instalação e seleção de dados}: Após instalar o plugin no QGIS, o usuário seleciona uma camada raster (GeoTIFF) que já está aberta no projeto, eliminando a necessidade de navegar novamente pelo sistema de arquivos.
    
    \item \textbf{Configuração e processamento}: Através de uma interface gráfica simplificada, o usuário define o arquivo de saída (.raw). O plugin processa automaticamente o dado de entrada, realizando as seguintes operações:
    \begin{itemize}
        \item Recorte e ajuste de resolução compatível com Unity
        \item Normalização para 16-bit sem sinal (unsigned integer), preservando a faixa de elevação original
        \item Detecção e exclusão automática de valores de padding/NoData
        \item Cálculo correto das dimensões (X, Y, Z) considerando diferenças entre sistemas de coordenadas geográficos e projetados
    \end{itemize}
    
    \item \textbf{Feedback e logs}: O plugin apresenta logs detalhados da conversão, incluindo informações sobre o processamento realizado e parâmetros sugeridos para importação no Unity.
    
    \item \textbf{Visualização no Unity}: O usuário importa o arquivo .raw gerado no Unity, utilizando as configurações sugeridas pelo plugin, e visualiza o terreno 3D resultante.
\end{enumerate}

\begin{figure}[htbp]
    \centering
    \includegraphics[width=0.8\textwidth]{poster/0_plugin_info}
    \caption{Fluxo de trabalho da prova de conceito - Etapa 1: Plugin instalado. Após instalar o plugin, o usuário seleciona camada raster no QGIS.}
    \label{fig:prova-1}
\end{figure}

\begin{figure}[htbp]
    \centering
    \includegraphics[width=0.8\textwidth]{poster/1_geotiff}
    \caption{Fluxo de trabalho da prova de conceito - Etapa 2: MDE em GeoTIFF. O usuário trabalha com dados geoespaciais já carregados no projeto.}
    \label{fig:prova-2}
\end{figure}

\begin{figure}[htbp]
    \centering
    \includegraphics[width=0.8\textwidth]{poster/2_plugin_dialog}
    \caption{Fluxo de trabalho da prova de conceito - Etapa 3: Processamento. Através de interface simplificada, o usuário define o arquivo de saída.}
    \label{fig:prova-3}
\end{figure}

\begin{figure}[htbp]
    \centering
    \includegraphics[width=0.8\textwidth]{poster/3_logs}
    \caption{Fluxo de trabalho da prova de conceito - Etapa 4: Logs de conversão. O plugin processa automaticamente (recorte, normalização 16-bit, detecção de padding, cálculo de dimensões), gera arquivo .raw compatível e apresenta logs.}
    \label{fig:prova-4}
\end{figure}

\begin{figure}[htbp]
    \centering
    \includegraphics[width=0.8\textwidth]{poster/4_terrain}
    \caption{Fluxo de trabalho da prova de conceito - Etapa 5: Terreno 3D (Unity). O usuário importa o arquivo .raw no Unity com as configurações sugeridas e visualiza o terreno 3D.}
    \label{fig:prova-5}
\end{figure}

A prova de conceito valida que a solução proposta atende aos requisitos funcionais estabelecidos, demonstrando que é possível automatizar o processo de conversão de MDEs do formato GeoTIFF para o formato RAW compatível com Unity, reduzindo significativamente a barreira técnica para pesquisadores não especializados em computação.

